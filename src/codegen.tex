Our code generation strategy follows a pattern based approach for fused sections,
whereas normal (non-fused) code calls optimized library functions. First, each fused section
is associated with a fusion node, the nodes are optimized, and lastly the code
is emitted.

%%% In this section we first introduce fusion node which collects the meta
%%% information of a fusible section.  Then, we present our implementation
%%% for constructing fusion nodes.  Finally, we generate code from fusion nodes.

\subsection{Fusion Nodes}

Each fused section is associated with a fusion node, a collection of metadata used
for generating the loop. For each section, we traverse the statements and collect:
(1) loop bounds; (2) fused expressions; (3) boolean mask (if any); and (4) reduction
operation (if any). The set of properties determines which code generation pattern
is used.

%% In this section we introduce fusion node for collecting the meta information of
%% a fusible section, and the code generation from fusion nodes.

%% One fusible section has one fusion node.
%% A fusion node needs to record the basic meta information which can be used to
%% construct a valid loop.  Thus, it has to traverse from the start statement of
%% the fusible section up to its predecessors until the last statement in the
%% fusible section.

%% During the traverse, the node keeps the loop head information including the size of
%% the loop, the condition mask if applicable, and whether it is a section for lists.
%% Meanwhile, it has to scan statement expressions to generate corresponding
%% expressions for both left-hand and right-hand side expressions by given type
%% and shape information.
%%% Q: Do we need to explain more?

%% \begin{figure}[htbp]
%% \begin{verbatim}
%%     {
%%       "loop_head": {
%%         "loop_size"   : string,
%%         "loop_cond"   : string,
%%         "loop_isList" : boolean
%%       },
%%       "loop_body": {
%%         "expr_lhs" : string,
%%         "expr_rhs" : string,
%%         "expr_tmp" : string[]
%%       },
%%       "start_stmt": stmt
%%     }
%% \end{verbatim}
%% \caption{Components of a simple loop for fusion.}
%% \label{fig:fusion_node_json}
%% \end{figure}

%%% A fusion node consists of a set of the following fields to save the meta
%%% information for a fusible section.

%%% \begin{description}
%%% \item[\texttt{loop_head}] describes the head of loop iteration including loop
%%% size, condition, and check for lists.
%%% 
%%% \begin{itemize}
%%% \item $loop\_size$: the number of loop iterations
%%% \item $loop\_cond$: an if-condition inside the loop
%%% \item $loop\_isList$: whether the loop is for list or vector
%%% \end{itemize}
%%% 
%%% \item[\texttt{loop_body}] represents the main computation of the loop which
%%% contains the following components.
%%% 
%%% \begin{itemize}
%%% \item $expr\_lhs$: left hand side expression 
%%% \item $expr\_rhs$: right hand side expression
%%% \item $expr\_tmp$: a list of temporary results
%%% \item $is\_r$: whether the start statement has a reduction operation
%%% \end{itemize}
%%% 
%%% \end{description}

%%% Note that the value of $loop\_cond$ is allowed to be empty that means there is
%%% no loop condition involved.
%%% 
%%% \subsection{Node Information Retrieval}
%%% 
%%% With a given fusion section, we fetch the information of its fusion node as described in
%%% \refAlgo{fetch_info}.
%%% 
%%% \todo{The algorithm has been commented out because it is hard to make it a
%%% short algorithm (too verbose). Will discuss with Alex to make this part clear.}

%%% \begin{lstlisting}
%%% entry:
%%%     Input: Given a fusion section (section) with a start statement (start)
%%%     Output: A fusion node
%%%     initialize a fusion node n
%%%     if (start is raze)
%%%         genNodeList(start, section)
%%%     else
%%%         genNodeVector(start, section)
%%% 
%%%     genNodeVector(stmt, section){
%%%         n.loop_size = genSize(stmt)
%%%         n.loop_isList = false
%%%         if(hasScanShape(stmt))
%%%             n.loop_cond = genNodeVector(getScanAlias(stmt),section)
%%%         else
%%%             n.loop_cond = ""
%%%         if(isReduction(stmt)) {
%%%             n.is_r = true
%%%             n.expr_lhs = genRop(stmt)
%%%             setVisited(stmt)
%%%             genNodeVectorRHS(getDefStmts(stmt)[0])
%%%         }
%%%         else {
%%%             n.is_r = false
%%%             n.expr_lhs = genTargetWithIndex(stmt)
%%%             n.expr_rhs = genNodeVectorRHS(stmt)
%%%         }
%%%     }
%%% 
%%%     genNodeVectorRHS(stmt, section){
%%%         setVisited(stmt)
%%%         def = getDefStmt(stmt)
%%%         if (def \in section){
%%%             if (map contains the def)
%%%                 return getOpMacro(stmt)+"("+map.get(def)+")"
%%%             if(hasOneChild(def))
%%%                 return getOpMacro(stmt)+"("+genNodeVectorRHS(def, section)+")"
%%%             else {
%%%                 tid = genTempId()    // a unique id
%%%                 tmp = tid + "=" + genNodeVectorRHS(def)
%%%                 map.insert(def, tid)
%%%                 expr_tmp.add(tmp)
%%%             }
%%%         }
%%%         else
%%%             return getOpMacro(stmt)+"("+genVectorExpr(stmt)+")"
%%%     }
%%% 
%%%     genNodeList(start, section){
%%%     }
%%% 
%%% \end{lstlisting}
 

%%% \begin{description}
%%% \item[Step 0:]
%%%     Determine if it is a fusion section for list, if true, goto Step X
%%% \item[Step 1:]
%%%     Identify the target (i.e. \texttt{expr_lhs}) which is stored in the start statement:
%%%     if a reduction operation is found, a scalar return is generated.
%%% \item[Step 2:]
%%% \end{description}

%%% \headstep{Collecting loop head.}

%%% \headstep{Collecting loop expression.}

\subsection{Code Generation for Vectors} \label{SubSec:CodeGen}

%% \begin{verbatim}
%% -------------------------------------------
%% Start stmt | Has Scan? | C Code
%% -------------------------------------------
%%    R       |  yes      |  ...
%%            |  no       |  ...
%% -------------------------------------------
%%    No R    |  yes      |  ...
%%            |  no       |  ...
%% -------------------------------------------
%% \end{verbatim}

\begin{figure}[htbp]
\par\noindent\rule{\columnwidth}{0.6pt}
% first row
\begin{subfigure}[t]{.45\columnwidth}
\begin{small}
\begin{verbatim}
// reduction: YES
// scan: YES
for(i=0; i<len; i++){
  if(cond[i]){
      z = z Rop expr_rhs;
}}
z = Rfinal(z);
\end{verbatim}
\end{small}
\caption{Case 0} \label{fig:codegen_c0}
\end{subfigure}
\hspace{1mm}
\rulesep
\hspace{2mm}
\begin{subfigure}[t]{.45\columnwidth}
\begin{small}
\begin{verbatim}
// reduction: YES
// scan: NO
for(i=0; i<len; i++){

  z = z Rop expr_rhs;
}
z = Rfinal(z);
\end{verbatim}
\end{small}
\caption{Case 1} \label{fig:codegen_c1}
\end{subfigure}
%\vspace{1mm}
\par\noindent\rule{\columnwidth}{0.6pt}
% second row
\begin{subfigure}[t]{.45\columnwidth}
\begin{small}
\begin{verbatim}
// reduction: NO
// scan: YES
c = 0;
for(i=0; i<len; i++){
  if(cond[i]){
    z[c++] = expr_rhs;
}}
\end{verbatim}
\end{small}
\caption{Case 2} \label{fig:codegen_c2}
\end{subfigure}
\hspace{1mm}
\rulesep
\hspace{2mm}
\begin{subfigure}[t]{.45\columnwidth}
\begin{small}
\begin{verbatim}
// reduction: NO
// scan: NO

for(i=0; i<len; i++){

  z[i] = expr_rhs;
}
\end{verbatim}
\end{small}
\caption{Case 3} \label{fig:codegen_c3}
\end{subfigure}
%\vspace{1mm}
\par\noindent\rule{\columnwidth}{0.6pt}
\caption{Code generation for vectors. \texttt{Rop}: reduction operation; \texttt{Rfinal}: final reduction step (e.g. divide by element count); \texttt{z}: accumulator/output vector.} \label{fig:codegen_vectors}
\end{figure}

Code generation for fused vector operations follows 4 patterns depending on the presence
of reduction and boolean selection. Each iterates over the length of the list, fuses the
RHS expressions, and produces the appropriate output. Reduction nodes accumulate a scalar
value, while non-reduction nodes create a new vector. \texttt{Rfinal} performs the final
step of the reduction (e.g. dividing by the number of elements to compute an average).
In the case of compression, the condition is first evaluated and the RHS computed if
necessary. \refFig{fig:codegen_vectors} shows the variations of the code generation patterns.

Note that when generating parallel code for \refFig{fig:codegen_c2}, we employ a
strategy that:
(1) counts the number of true elements in each segment and computes an offset for
each segment; and
(2) divides the boolean vector into segments evenly based on the number of cores.
Each thread thus maintains a segment of the boolean vector independently. For all other
cases, typical parallel strategies are effective.

%%% \begin{algorithm}
%%% \SetAlgoLined
%%% \KwResult{Result}
%%% let op $\gets$ op\_name\;
%%% let x,y $\gets$ op\_args\;
%%% \Switch{op_kind}{
%%%   \Case{Unary Elementwise with E(op,x)}{
%%%     \Return{isScalar(x) ? op(x) : op($\vec{x}$)}\;
%%%   }
%%%   \Case{Binary Elementwise with E(op,x,y)}{
%%%     \eIf{isScalar(x)}{
%%%       \Return{isScalar(y) ? op(x,y) : op(x,$\vec{y}$)}\;
%%%     }{
%%%       \Return{isScalar(y) ? op($\vec{x}$,y) : op($\vec{x}$,$\vec{y}$)}\;
%%%     }
%%%   }
%%%   \Case{Reduction with R(op,x)}{
%%%     \Return{isScalar(x) ? op(=,x) : op(=,$\vec{x}$)}\;
%%%   }
%%%   \Case{Scan with S(m, x)}{
%%%     \Return{$\vec{x}$}\;
%%%   }
%%%   \Case{Indexing with $X$(x,y)}{
%%%     \Return{x($\vec{y}$)}\;
%%%   }
%%%   \Case{Special boolean with B(x,y) and op_name as func}{
%%%     \Return{isScalar(x) ? func(x,y) : func(x,$\vec{y}$)}\;
%%%   }
%%% }
%%% \caption{Code Generation for Sub-expressions.} \label{algo:expr}
%%% \end{algorithm}

\subsection{Generating Code for Lists}

%%% \begin{figure}[htbp]
%%% \par\noindent\rule{\columnwidth}{0.6pt}
%%% % first row
%%% \begin{subfigure}[t]{.45\columnwidth}
%%% \begin{verbatim}
%%% for(i=0;i<len;i++){
%%%   c = list[i]; // cell
%%%   // init t
%%%   for(j=0;j<c_len;j++){
%%%       t = t Rop(cell[j])
%%%   }
%%%   z[i] = Rfinal(t);
%%% }
%%% \end{verbatim}
%%% \caption{Code generation template.} \label{fig:codegen_list_temp}
%%% \end{subfigure}
%%% \hspace{1mm}
%%% \rulesep
%%% \hspace{2mm}
%%% \begin{subfigure}[t]{.45\columnwidth}
%%% \begin{verbatim}
%%% for(i=0;i<len;i++){
%%%   c = v[i]; // cell
%%%   t = 0;  // init for sum
%%%   for(j=0;j<c_len;j++){
%%%       t = t + age[c[i]];
%%%   }
%%%   avg[i] = t/c_len;
%%% }
%%% \end{verbatim}
%%% \caption{Example code for \refFig{fig:list_horseir}} \label{fig:codegen_list_example}
%%% \end{subfigure}
%%% \vspace{1mm}
%%% \par\noindent\rule{\columnwidth}{0.6pt}
%%% \caption{Code generation for lists.} \label{fig:codegen_lists}
%%% \end{figure}

\begin{figure}[htbp]
\par\noindent\rule{\columnwidth}{0.6pt}
%[xleftmargin=.2\columnwidth]
\begin{small}
\begin{verbatim} 
for(i=0; i<list.len; i++){ /* loop over cells */
  cell = list[i]; /* fetch one cell */
  /* init t */
  for(j=0; j<cell.len; j++){ /* loop over content */
      t = t Rop (cell[j])
  }
  z[i] = Rfinal(t); /* store final value */
}
\end{verbatim}
\end{small}
\vspace{-1mm}
\par\noindent\rule{\columnwidth}{0.6pt}
\caption{Code generation for lists. \texttt{Rop}: reduction operation; \texttt{Rfinal}: final reduction step (e.g. divide by element count); \texttt{t}: cell accumulator; \texttt{z}: output vector.} \label{fig:codegen_lists}
\end{figure}

List fusion nodes compute a single value per list cell and return a vector. \refFig{fig:codegen_lists} shows the code generation pattern for lists. As seen in
the figure, there are two loops present: an outer loop iterating over cells and an
inner loop computing the reduction expression for each cell.

% \refFig{fig:codegen_list_example} is an example code in C for
% \refFig{fig:list_horseir}.

Note that the ratio of \texttt{list.len} and \texttt{cell.len} may vary greatly.
When parallel code is generated, we may therefore parallelize the outer or inner loop
depending on the data. In our implementation, we use a simple runtime heuristic based
on the size ratio to choose which loop runs in parallel.

\subsection{Further Fusion Opportunities}

Fusible sections can be further fused if:
(1) they share the same loop head, and
(2) there is no data dependence between the loop bodies.
This is particularly useful and common in column-based IMDBs where data is fetched
from multiple independent columns using a single array of indices (e.g. the result
of a join). It also improves parallelization as the number of synchronization barriers
is reduced.

%  A tuple of a code generation rule is
%  
%  \begin{center}
%  T = \tupleSingle{$op\_kind$}{$op\_name$}{$n$}{$cond$}{$expr$\{, $expr$\}}\\
%  \end{center}
%  
%  where
%  
%  \begin{itemize}
%  \item $op\_kind$: the kind of the operation;
%  \item $op\_name$: the name of the operation;
%  \item $n$:  a loop from 0 (inclusive) to n (exclusive);
%  \item $cond$: an if-condition inside the loop; and
%  \item $expr$\{, $expr$\}: a list of expressions as input for the main computation in the loop body;
%  \end{itemize}
%  
%  In the first example as follows, $E_2$ means the binary elementwise operation;
%  the condition is set to null; and the main computation attributes to the
%  addition of x[i] and y[i].  On the right side, we can see that its
%  corresponding code generated from the given tuple, where the loop is generated
%  with a range of [0,n) and the result of the addition is saved to a target
%  array.
%  
%  \vv
%  
%  % textwidth: two columns
%  
%  \noindent
%  \begin{minipage}{.53\columnwidth}
%  \begin{tabular}{l}
%  \textbf{Example 1:} \\
%  \tuple{$E_2$}{PLUS}{n}{null}{x[i]}{y[i]}
%  \end{tabular}
%  \end{minipage}
%  %\hfill
%  \begin{minipage}{.37\columnwidth}
%  \begin{tabular}{|c}
%  %\hspace{1mm}
%  \begin{minipage}{\columnwidth}
%  \begin{verbatim}
%  for(int i=0; i<n; i++){
%    targ[i] = x[i] + y[i];
%  }
%  \end{verbatim}
%  \end{minipage}
%  \end{tabular}
%  \end{minipage}
%  
%  \vspace{4mm}
%  
%  In the second example below, $R$ means the reduction; a condition is associated
%  with a variable $b$; and the expression is  a multiplication of \texttt{x[i]}
%  and \texttt{y[i]}. On the right side, we can see that the loop body contains a
%  boolean mask \texttt{b[i]} with the main computation expression inside the true
%  block.
%  
%  \vv
%  
%  \noindent
%  \begin{minipage}{.53\columnwidth}
%  \begin{tabular}{l}
%  \textbf{Example 2:} \\
%  \tupleSingle{R}{SUM}{$n$}{$b$}{mul(x[i],y[i])}
%  \end{tabular}
%  \end{minipage}
%  %\hfill
%  \begin{minipage}{.37\columnwidth}
%  \begin{tabular}{|c}
%  %\hspace{1mm}
%  \begin{minipage}{\columnwidth}
%  \begin{verbatim}
%  for(int i=0; i<n; i++){
%    if(b[i]){
%      targ += x[i] * y[i];
%    }
%  }
%  \end{verbatim}
%  \end{minipage}
%  \end{tabular}
%  \end{minipage}
%  
%  \vv
%  
%  \refTab{Tab:op_list} shows the information of classified primitive functions.  The number of operands varies so that the number of elements in a tuple can be different.
%  
%  \noindent
%  \begin{table}[htbp]
%  \centering
%  \caption{List of kinds of primitive functions} \label{Tab:op_list}
%  \begin{tabular}{|c|c|c|}
%  \hline
%  Kind     & Name               & \# of Operands \\ \hline
%  $E_1$    & Unary elementwise  & 1 \\ \hline
%  $E_2$    & Binary elementwise & 2 \\ \hline
%  $R$      & Reduction          & 1 \\ \hline
%  $C$      & Compression        & 2 \\ \hline
%  \end{tabular}
%  \end{table}
%  
%  \subsection{Elementwise Function - Dyadic}
%  
%  \refTab{Tab:gen_elem_dyadic} presents the shape rules of dynamic elementwise functions after the conformability analysis.  Let \TupleETwo{0} denote the first rule inside the table, then we can have all 9 cases with their corresponding code generation tuples in \refTab{Tab:gen_rules_elem_dyadic}.
%  
%  \begin{table}[htbp]
%  \centering
%  \caption{Elementwise Functions - Dyadic} \label{Tab:gen_elem_dyadic}
%  \vv
%  \begin{tabular}{c||c|c|c}
%  \hline
%  x op y & \scan{$b_0$}{$n_0$} & $m_0$ ($m_0 \neq 1$) & 1 \\
%  \hline \hline
%  \scan{$b_1$}{$n_1$} & \scan{$b_0$}{$n_0$} & Fail & \scan{$b_1$}{$n_1$} \\
%  \hline
%  $m_1$ ($m_1 \neq 1$) & Fail & $m_0$ & $m_1$ \\
%  \hline
%  1     & \scan{$b_0$}{$n_0$} & $m_0$ & 1 \\
%  \hline
%  \end{tabular}
%  \end{table}
%  
%  
%  \begin{table}[htbp]
%  \centering
%  \caption{Code Generation Rules for Dyadic Elementwise Functions } \label{Tab:gen_rules_elem_dyadic}
%  \vv
%  \begin{tabular}{c||c|c|c}
%  \hline
%  x op y & \scan{$b_0$}{$n_0$} & $m_0$ ($m_0 \neq 1$) & 1 \\
%  \hline \hline
%  \scan{$b_1$}{$n_1$} & \TupleETwo{0} & \TupleETwo{1} & \TupleETwo{2} \\
%  \hline
%  $m_1$ ($m_1 \neq 1$) & \TupleETwo{3} & \TupleETwo{4} & \TupleETwo{5} \\
%  \hline
%  1     & \TupleETwo{6} & \TupleETwo{7} & \TupleETwo{8} \\
%  \hline
%  \end{tabular}
%  \\ \vv
%  \begin{tabular}{|c|l|l|}
%  \hline
%  Case ID & \multicolumn{1}{c|}{Code Generation Tuple} & \multicolumn{1}{c|}{Output Expr} \\
%  \hline
%  \TupleETwo{0} & \tuple{$E_2$}{$op$}{$n_0$}{$b_0$}{$x[i]$}{$y[i]$} & $x[i]\ op\ y[i]$  \\ \hline
%  \TupleETwo{1} & ------ & ------ \\ \hline
%  \TupleETwo{2} & \tuple{$E_2$}{$op$}{$n_0$}{$b_0$}{$x$}{$y[i]$}    & $x\ op\ y[i]$     \\ \hline
%  \TupleETwo{3} & ------ & ------ \\ \hline
%  \TupleETwo{4} & \tuple{$E_2$}{$op$}{$n_0$}{$null$}{$x[i]$}{$y[i]$}& $x[i]\ op\ y[i]$  \\ \hline
%  \TupleETwo{5} & \tuple{$E_2$}{$op$}{$n_0$}{$null$}{$x$}{$y[i]$}   & $x\ op\ y[i]$  \\ \hline
%  \TupleETwo{6} & \tuple{$E_2$}{$op$}{$n_0$}{$b_0$}{$x[i]$}{$y$}    & $x[i]\ op\ y$  \\ \hline
%  \TupleETwo{7} & \tuple{$E_2$}{$op$}{$n_0$}{$null$}{$x[i]$}{$y$}   & $x[i]\ op\ y$  \\ \hline
%  \TupleETwo{8} & \tuple{$E_2$}{$op$}{$null$}{$null$}{$x[i]$}{$y$}  & $x[i]\ op\ y$  \\ \hline
%  \end{tabular}
%  \end{table}
%  
%  
%  %Row 1
%  %
%  %\begin{itemize}
%  %\item \TupleCase{\scan{$b_1$}{$n_1$}}{\scan{$b_0$}{$n_0$}} = \tuple{$E_2$}{$op$}{$n_0$}{$b_0$}{$x[i]$}{$y[i]$}
%  %\item \TupleCase{\scan{$b_1$}{$n_1$}}{$m_0$} = N/A
%  %\item \TupleCase{\scan{$b_1$}{$n_1$}}{1} = \tuple{$E_2$}{$op$}{$n_0$}{$b_0$}{$x$}{$y[i]$}
%  %\end{itemize}
%  %
%  %Row 2
%  %
%  %\begin{itemize}
%  %\item \TupleCase{$m_1$}{\scan{$b_0$}{$n_0$}} = N/A
%  %\item \TupleCase{$m_1$}{$m_0$} = \tuple{$E_2$}{$op$}{$n_0$}{$null$}{$x[i]$}{$y[i]$}
%  %\item \TupleCase{$m_1$}{1} = \tuple{$E_2$}{$op$}{$n_0$}{$null$}{$x$}{$y[i]$}
%  %\end{itemize}
%  %
%  %Row 3
%  %
%  %\begin{itemize}
%  %\item \TupleCase{1}{\scan{$b_0$}{$n_0$}} = \tuple{$E_2$}{$op$}{$n_0$}{$b_0$}{$x[i]$}{$y$}
%  %\item \TupleCase{1}{$m_0$} = \tuple{$E_2$}{$op$}{$n_0$}{$null$}{$x[i]$}{$y$}
%  %\item \TupleCase{1}{1} = \tuple{$E_2$}{$op$}{$null$}{$null$}{$x[i]$}{$y$}
%  %\end{itemize}
%  
%  \subsection{Elementwise Function - Monadic}
%  
%  \refTab{Tab:gen_elem_monadic} presents the shape rules of monadic elementwise functions after the conformability analysis.  Let \TupleEOne{0} denote the first rule inside the table, then we can have all 3 cases with their corresponding code generation tuples in \refTab{Tab:gen_rules_elem_monadic}.
%  
%  \begin{table}[ht!]
%  \centering
%  \caption{Elementwise Functions - Monadic} \label{Tab:gen_elem_monadic}
%  \vv
%  \begin{tabular}{c|c|c|c}
%  \hline
%  op x & \scan{$b_0$}{$n_0$} & $m_0$ ($m_0 \neq 1$) & 1 \\
%  \hline
%         & \scan{$b_0$}{$n_0$} & $m_0$ & 1 \\
%  \hline
%  \end{tabular}
%  \end{table}
%  
%  \begin{table}[ht!]
%  \centering
%  \caption{Code Generation Rules for  Monadic Elementwise Functions} \label{Tab:gen_rules_elem_monadic}
%  \vv
%  \begin{tabular}{c|c|c|c}
%  \hline
%  op x & \scan{$b_0$}{$n_0$} & $m_0$ ($m_0 \neq 1$) & 1 \\
%  \hline
%         & \TupleEOne{0} & \TupleEOne{1} & \TupleEOne{2} \\
%  \hline
%  \end{tabular}
%  \\ \vv
%  \begin{tabular}{|c|l|}
%  \hline
%  Case ID & \multicolumn{1}{c|}{Code Generation Tuple} \\
%  \hline
%  \TupleEOne{0} & \tupleSingle{$E_1$}{$op$}{$n_0$}{$b_0$}{$x[i]$}  \\ \hline
%  \TupleEOne{1} & \tupleSingle{$E_1$}{$op$}{$n_0$}{$null$}{$x[i]$} \\ \hline
%  \TupleEOne{2} & \tupleSingle{$E_1$}{$op$}{$null$}{$null$}{$x$}     \\ \hline
%  \end{tabular}
%  \end{table}
%  
%  \subsection{Reduction}
%  
%  An operation reduction can be performed the code below which sums up total value in an integer vector \texttt{x}.
%  
%  
%  \begin{minipage}{.4\columnwidth}
%  \begin{verbatim}
%  S1: sum = 0;
%  S2: for(int i=0; i<n; i++){
%  S3:     sum = sum + x[i];
%  S4: }
%  \end{verbatim}
%  \end{minipage}
%  \hfill
%  % https://tex.stackexchange.com/a/29237
%  \begin{minipage}{.4\columnwidth}
%  \small
%  \begin{tikzpicture}
%  \node[draw, align=left] at (0,0) {\baselineskip=20pt L0: sum_0 = 0;\\ L1: sum_1 = sum_0 + x[0]; \\ L2: sum_2 = sum_1 + x[1]; \\ L3: sum_3 = sum_2 + x[2]; \\ ...};
%  \draw [->, densely dotted, blue, line width=1pt] (-0.8,0.6)   -- (0.1,0.45);
%  \draw [->, densely dotted, blue, line width=1pt] (-0.8,0.25)  -- (0.1,0.1);
%  \draw [->, densely dotted, blue, line width=1pt] (-0.8,-0.15) -- (0.1,-0.3);
%  \draw [->, densely dotted, blue, line width=1pt] (-0.8,-0.5)  -- (0.1,-0.65);
%  \end{tikzpicture}
%  \end{minipage}
%  \begin{minipage}{.4\columnwidth}
%  \end{minipage}
%  
%  \vv
%  
%  The variable \texttt{sum} in the statement \texttt{S3} has the data dependency
%  on the previous value of \texttt{sum}.  Therefore, fusing primitive functions
%  after the reduction is unfeasible.
%  
%  
%  \begin{table}[ht!]
%  \centering
%  \caption{Reduction} \label{Tab:gen_reduction}
%  \vv
%  \begin{tabular}{c|c|c|c}
%  \hline
%  op x & \scan{$b_0$}{$n_0$} & $m_0$ ($m_0 \neq 1$) & 1 \\
%  \hline
%         & 1 & 1 & 1 \\
%  \hline
%  \end{tabular}
%  \end{table}
%  
%  \begin{table}[ht!]
%  \centering
%  \caption{Code Generation Rules for Reduction} \label{Tab:gen_rules_reduction}
%  \vv
%  \begin{tabular}{c|c|c|c}
%  \hline
%  op x & \scan{$b_0$}{$n_0$} & $m_0$ ($m_0 \neq 1$) & 1 \\
%  \hline
%         & \TupleEOne{0} & \TupleEOne{1} & \TupleEOne{2} \\
%  \hline
%  \end{tabular}
%  \\ \vv
%  \begin{tabular}{|c|l|}
%  \hline
%  Case ID & \multicolumn{1}{c|}{Code Generation Tuple} \\
%  \hline
%  \TupleEOne{0} & \tupleSingle{$R$}{$op$}{$n_0$}{$b_0$}{$x[i]$}  \\ \hline
%  \TupleEOne{1} & \tupleSingle{$R$}{$op$}{$n_0$}{$null$}{$x[i]$} \\ \hline
%  \TupleEOne{2} & \tupleSingle{$R$}{$op$}{$null$}{$null$}{$x$}     \\ \hline
%  \end{tabular}
%  \end{table}


