While database systems have traditionally stored relational tables on a per-row
basis, recent developments have shown that a column-store format can outperform
traditional RDBMS for many workloads, particularly when the data fits into
main memory.
One of the first research prototypes of a column-based database manage systems
(DBMS) was MonetDB \cite{IdreosS2012}, still an active research project, but
many commercial systems now also support column-based storage (e.g.
Microsoft SQL Server~\cite{msqlserver},
SAP HANA~\cite{FarberF2012}, and
HyPer~\cite{Neumann2011:HyPer}).
% Microsoft SQL server
This fundamental shift in database architecture has resulted in a large body of
research efforts within the database community looking into efficient SQL query
execution on such a data format.

Given that in a column store each column of a table is stored as an array,
array programming languages appear as a promising construct to implement query
execution. In fact, \OldPaperAuthor introduced an intermediate representation (IR),
\textit{HorseIR}, providing efficient database operators, while also serving as
a general purpose language.  This extends
the approach more commonly used to translate from relational algebra to optimized code (as in 
Hyper~\cite{Neumann2011:HyPer}), allowing integration of 
operators or code from other array-based languages, such as MATLAB.
SQL queries are first transformed into execution plans using standard database
optimization techniques that describe in which order and how database operators
such as joins, selections and aggregations should be executed.  Once translated
into HorseIR, the resulting code can be optimized and compiled into multi-platform
C code.

An important optimization in this context is aimed at reducing the number of intermediate
computations.
For example, it is common for SQL queries to do a selection (selecting some
rows based on the filter criteria on one column) and then perform aggregation,
such as a summation.  In an array-based language this can be represented as

\begin{small}
\begin{Verbatim}[xleftmargin=.3\columnwidth]
R = sum(filter(C, A))
\end{Verbatim}
\end{small}

\noindent{}where the condition \texttt{C} is a boolean vector, the input array \texttt{A}
is a vector with the same number of elements, and the result \texttt{R} is a
scalar.  A straightforward way to generate C code from this is to generate two loops, the first applying the boolean selection and producing an intermediate result, and the second loop performing the reduction on this intermediate result.  A more optimized version, however, would fuse the two loops into one, performing the reduction on the fly and avoiding generation of an intermediate result:

\begin{small}
\begin{Verbatim}[xleftmargin=.15\columnwidth]
R=0;
for(int i=0; i<C.size; i++){
    if(C[i]) { R += A[i]; } // true block
}
\end{Verbatim}
\end{small}

\noindent{}The analysis required to generate such code, however, is complex,
depending on non-trivial array dependency analysis, the specific operators or built-in functions, as well as knowledge of array dimensions.
This complexity tends to prevent such optimization opportunities from being discovered in the generated C code.
HorseIR attempts to solve this by using a pattern-based approach, but this is also limited, requiring
considerable knowledge of database operators, and associated manual tuning.

\begin{figure}[htbp]
\centering
\includegraphics[width=.9\columnwidth]{./src/figure/basic-idea.pdf}
\caption{Automatic operator fusion for SQL queries includes element-wise fusion and a small set of patterns.}
\label{fig:fusion_idea}
\end{figure}

In this paper we provide a systematic approach to analyze the SQL query programs
and generate optimized code. As shown in \refFig{fig:fusion_idea},
\textit{automatic fusion} is a blend of smart element-wise fusion and a limited set of
patterns for cases that cannot be handled. More specifically, we provide a detailed
analysis of the most important built-in functions used for executing SQL operators
and explain how to optimize across such functions. Our approach is based on a
\textit{shape analysis}, which describes the layout and size of variables as
functions are applied. This allows us to fuse loops extensively, and avoid unnecessary
iterations over the data and intermediate results.  

% contribution
The contributions of this paper are summarized as follows:

\begin{itemize}
\item We explore a new approach to improve database query performance by
      applying techniques derived from array programming. We focus on HorseIR, an IR specifically designed to support database query execution.
\item We perform shape analysis on the built-in functions of HorseIR that represent database operators. This allows us to collect precise shape information and provide a conformability analysis to identify fusible sections. 
\item We present a set of optimization and code generation strategies to automatically
      generate optimized code.
\item We conduct experiments on the database benchmark TPC-H, to show the effectiveness of our technique.
\end{itemize}
