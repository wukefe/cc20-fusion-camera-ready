% \subsection{Fusion Teachniques}
% \subsection{Related Database Systems}

% \begin{description}
% \item[Lift with patterns] High-level synthesis of functional patterns with
%     Lift~\cite{Kristien19:LiftPatterns}.
% \item[WeldIR] Evaluating End-to-End Optimization for Data Analytics
%     Applications in Weld~\cite{Palkar18:Weld}.
% \item[Peloton] Our code generation fits the SIMD predicate evaluation used in
%     Peloton. That implies our code generation can be further improved with
%     lower but efficient instructions.
% \end{description}

% \todo{Add "How to Architect a Query Compiler, Revisited"~\cite{TahboubER18:Revisited}}
% \todo{Add "Generating code for holistic query evaluation"~\cite{}}

Fusion techniques are popular due to their success in reducing intermediate
results. This is especially true for compiled environments which easily
allow code fusion.

\head{Fusion in programming languages}

% Operator fusion is an effective technique for high-level programming languages,
% especially for array-based programming languages.

In the MATLAB-to-FORTRAN compiler~\cite{rose1999:techniques}, shape and size
information can be obtained from a conformability analysis with a set of
well-defined conformable operators for scalar, vector, and matrix.
However, no further operator fusions after conformability analysis are provided.

In the R programming language, a vectorizer is proposed for the built-in
function \texttt{Apply}~\cite{Wang2015:vectorization}. The function
\texttt{Apply} is similar to our list-based functions, which take
a function and a list of data inputs as parameters and repeatedly apply
the function. Their vectorizer involves both code and data transformation
while we focus on code optimization. Additionally, we explore a wider set
of built-in functions rather than only a single list-based function.

Ju et al.~\cite{ju1994:array} investigate built-in function fusion in a pure
array-based programming language, APL. They classify functions into groups
based on the features and present a model to help generate
parallel code. Ching et al.~\cite{ChingZ12parallel}
use simple techniques to fuse arithmetic functions when compiling APL to parallel
C code. In contrast, our work is aimed at array-based programs generated from
SQL queries, considering functions important to database operators.

Similar to optimizations for arrays~\cite{KumarH14,FoleyH16matjuice},
we exploit type and shape information of arrays to generate efficient code.
But due to the high-level semantics of HorseIR programs, we avoid
complicated vectorization techniques~\cite{menon1999case,ChenKLH16}.

\head{Fusion in database query optimizations}

%As the previal of query compilers in database systems, it is important to
%exploit fusion opportunities in standard SQL queries in order to generate
%efficient code.

% fusing multiple relational operations.
%HIQUE~\cite{Krikellas10:HIQUE} generates C++ code using code templates for each operator.
% As can be seen, they share our goal of fusing operators. However, we approach
% the problem from a static analysis perspective that operates on the IR-level
% rather than directly on the execution plan.

\new{
Loop fusion for query programs is typically achieved through a fixed set of
complex predefined set of rules.
The DBLAB/LB query compiler~\cite{ShaikhhaKPBD016} provides rules for
different loop fusion algorithms to generate optimized code
\cite{Shaikhha2018:LoopFusion}.
WeldIR~\cite{Palkar18:Weld} adopts rule-based optimizations for element-wise
and common-loop-head fusion.
HorseQC~\cite{FunkeBNMT18:HorseQC} and HIQUE~\cite{Krikellas10:HIQUE} present
rule-based fusion strategies.  
However, just as the patterns in \OldPaper, rules are challenging to generalize. % page 21
}

\new{
HyPer~\cite{Neumann2011:HyPer}, an in-memory database system, adopts a data-centric
model when compiling SQL queries to LLVM code. They developed a greedy
algorithm to produce/consume operators directly in an execution plan and fuse relational
operations directly.
Peloton~\cite{Menon2017:Peloton} considers operator fusion for operators within a pipeline.
The stream-fusion~\cite{Shaikhha2018:LoopFusion} needs to define extra
fusion-related constructs for collections.
% producing, consuming, unfolding, and destroying a collection.
By contrast, our fusion strategy is a systematic approach which
collects precise shape information on a well-defined array-based language for
automatic fusion.
}

%%% The Peloton in-memory DBMS~\cite{Menon2017:Peloton} considers operator
%%% fusion for operators within a pipeline. Again, the use quite specialized fusion
%%% rules, while we provide a more general optimization approach.
%%% On the other hand, their core strategy, SIMD predicate evaluation, can
%%% be applied to our code generation directly.

\head{Performance-oriented systems}

%%% Weld~\cite{Palkar18:Weld} is designed for optimizing cross different platforms
%%% with an intermediate representation WeldIR. With rule-based fusion, it is able
%%% to perform element-wise fusion and common-loop-head fusion.
TVM~\cite{Chen18:TVM}, a system designed for deep learning, introduces
operator fusion for graph operators when generating efficient GPU code.
Their approach is limited, as the fusion rules are fairly simple and
specific to fusion between groups.
We provide a more sophisticated fusion approach which considers more groups
and defines systematic data-flow analysis to identify fusion opportunities.

LIFT~\cite{Kristien19:LiftPatterns} shows a different approach by defining
complex functional patterns with precise descriptions for generating efficient
code for parallel devices such as FPGA, as its input code from various domains
rather than database queries.

% \todo{Q: Do we need to mention some papers/tools/systems which do not support
% operator fusion? for example, MonetDB,HyPer,...}


