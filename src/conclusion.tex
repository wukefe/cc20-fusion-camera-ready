Array-based languages are a natural fit for column-oriented approaches to database systems.  With a straightforward code generation strategy, however, benefits are reduced by the need to store intermediate computations, and this is not easily compensated for by optimizations normally applied in the low-level, generated code.  Our work exploits a higher level intermediate representation (HorseIR) to demonstrate that a well-informed, methodical optimization approach can help identify loop fusion opportunities, allowing the multiple steps implicit in query execution to be efficiently aggregated into single loops with composite operations.  This greatly improves performance over more naive query-code generation, both in single and multi-processor execution contexts.

Future work is aimed at expanding the set of strategies we have for fusing array-based operations.  Much of our current design remains conservative, and dynamic specialization or more complex, in-loop control flow would potentially allow fusing statements with different shapes.  We are also interested in optimizing compile time, which would enable our approach to be more easily applied to runtime query generation.
